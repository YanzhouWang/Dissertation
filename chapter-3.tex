\chapter{Needle Manipulation Modeling and Simulation} \label{chap:chap-3}


\section{Introduction}
\label{sec:chap_3_introduction}

% For percutaneous needle interventions, a thin needle needs to penetrate multiple layers of soft human tissues before reaching the target. Although modern medical robots are great at aligning the needle with the entry point, initial needle-to-skin contact can cause skin sliding prior to needle penetration, which leads to an immediate path error at the entry point. If the pre-operative needle path was to be followed, \emph{in situ} needle manipulation is unavoidable.

% Although there are ways to mitigate the issue -- adding a small incision on the entry point to ease needle penetration, or fixing the skin surface to prevent sliding motion -- there are other sources of error from patient-robot registration that are not completely avoidable, thus making the aforementioned error mitigation strategy ineffective.

% In actual clinical practice, it is not uncommon for the surgeon to miss the targets on the first try. In
% fact, by using intraoperative medical imaging and hand-eye coordination from experience, surgeons repeatedly perform \emph{in situ} needle manipulations by bending the needle base in order to iteratively reduce needle placement error.

During percutaneous needle interventions, a thin needle needs to penetrate multiple layers of soft tissues before reaching the anatomical target. For example, during a spinal injection procedure, the surgeon advances a spinal needle and injects diagnostic or therapeutic agent once the needle tip reaches the anatomical target. Spinal needles in common use today made from stainless steel alloys and are sized 20 gauge and above (1mm or less in diameter), making them less flexurally rigid and more susceptible to path deflections~\parencite{calthorpe2004history,silbergleit2001imaging,tsen2006needles}.

It is not uncommon for the surgeon to miss the target on their first try even with intraoperative imaging, such as fluoroscopy and computed tomography (CT). For example, initial needle-to-skin contact can cause skin sliding prior to needle penetration, which leads to an immediate path error at the needle entry point; inherent needle flexibility and unmodeled interaction with soft tissue can further deviate the needle from the planned path. Therefore, \emph{in situ} needle manipulation using intraoperative imaging guidance is often required to correct the placement error.

\section{In Situ Needle Base Manipulation}
\label{sec:in_situ_needle_base_manipulation}

\section{Finite Element Formulation and Solution}
\label{sec:finite_element_formulation_and_solution}

\section{Needle Insertion Simulation and Treatment of Generic Control Inputs}
\label{sec:simulation_and_inputs}


\section{Real-time Simulation}
\label{sec:realtime_simulation}


%%% Local Variables:
%%% mode: LaTeX
%%% TeX-master: "main"
%%% End:
