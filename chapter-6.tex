\chapter{Conclusion and Future Directions}
\label{chap:chap-6}

In this dissertation, we present our approach for autonomous flexible needle manipulations for percutaneous interventions without limitation of specific imaging modality. Inspired from freehand needle manipulations in clinical practice, this work takes a model-first approach to understand the intricate interactions between a flexible needle and surrounding soft tissues, and control the needle so that targeting can be precise and minimally invasive.

After an in-depth analysis of the role of kinematic remote center of motion (RCM) constraint in percutaneous needle interventions \added[id=ii]{in~\cref{chap:chap-2},} we propose \added[id=ii]{in~\cref{chap:chap-3}} a needle-tissue interaction model that captures multilayered soft tissue environment during \textit{in situ} needle adjustment, recreates strain-hardening soft tissue behaviors, and allows multi-modal and image-agnostic feedback for needle shape reconstruction. We extend the model \added[id=ii]{in~\cref{chap:chap-4}} to create a simulation that generates the model solution in realtime using nonlinear finite element method, captures a variety of needle tip geometries, and allows generic needle shape manipulation inputs that go beyond base manipulation. We then conduct thorough analysis on simulation-driven control and feedback methods, particularly resolved-rate control for base manipulation with symmetric-tip needles, and a comparison of Broyden's update versus finite difference methods for obtaining a system Jacobian to manipulate the shape of bevel-tip needles. We finish \added[id=ii]{in~\cref{chap:chap-5}} by presenting sampling-based trajectory generating and tracking methods, and verify our proposed approach using both plastisol and \textit{ex viv} soft tissues.

Throughout this work, our model reside in the plane of adjustment, and tissue deformation is not explicitly solved using finite element method. Our design choice is particularly influenced by our early focus on MRI-guided spinal injections, during which the spinal target does not move and tissue motion cannot be efficiently tracked. Here, we would like to highlight some additional considerations when it comes to modeling, simulation, and control of flexible needles:

Model complexity, speed of solution, and feedback modality go hand-in-hand. Complex models tend to take longer to solve, which might hinder time-sensitive decision making during insertion. Yet simple models require higher-fidelity feedback to account for model deficiencies. For example, coupling a ``virtual spring'' model with fluoroscopy can achieve fast and accurate needle insertion, yet the cost of radiation is high; a more complex needle-tissue interaction model can achieve a higher simulation fidelity, but requires more computational overhead to obtain a solution. High-fidelity models often need to undergo model reduction to be suitable for control, therefore there needs to be a balance between model complexity and rate of solution to ensure a timely and good sim-to-real transfer.

Some simulators take target motion into account. This is useful for offline simulation, and possibly online feedforward control. However, due to the inevitable sim-to-real gap, questions remain whether the target motion can be effectively tracked, and whether the online controller can adjust to account for such discrepancy. In our current work, we do not rely on medical imaging, therefore inherently limits our ability to track target motion; however, as discussed in~\cref{sec:chap-5-results-and-discussion}, if target state can be tracked, we can augment the system state so that the same control framework can be used.

In this work, our attention has been focused on fully autonomous needle control, with the intention to push the boundary of surgical automation and empower future generations of clinicians. However, we recognize the importance of human guidance, which inspired this work in the first place. It is only by gaining their approval and trust can we develop robust cooperation between surgeons and robots, and bring beneficial changes to the healthcare landscape.

%%% Local Variables:
%%% mode: latex
%%% TeX-master: "main"
%%% End:
