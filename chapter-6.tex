\chapter{Conclusion}
\label{chap:chap-6}

In this dissertation, we present our approach for autonomous flexible needle manipulations for percutaneous interventions without limitation of specific imaging modality. Inspired from freehand needle manipulations in clinical practice, this work takes a model-first approach to understand the intricate interactions between a flexible needle and surrounding soft tissues, and control the needle so that targeting can be precise and minimally invasive.

After an in-depth analysis of the role of kinematic remote center of motion (RCM) constraint in percutaneous needle interventions \added[id=ii]{in~\cref{chap:chap-2},} \added[id=rt]{we point out the fact that while a robot RCM is useful for needle \emph{adjustment}, needle \emph{insertion} complicates its practical value due to the curved needle trajectory after such adjustment. In addition, presence of a kinematic RCM in percutaneous needle intervention does not guarantee minimal invasiveness as opposed to laparoscopic surgeries, since the inserted part of the needle is completely surrounded by soft tissues and needle can deform upon external forces and torques. To gain a better understanding of needle manipulation, we} propose \added[id=ii]{in~\cref{chap:chap-3}} a needle-tissue interaction model that captures multilayered soft tissue environment during \textit{in situ} needle adjustment, recreates strain-hardening soft tissue behaviors, and allows multi-modal and image-agnostic feedback for needle shape reconstruction. We extend the model \added[id=ii]{in~\cref{chap:chap-4}} to create a simulation that generates the model solution in realtime using nonlinear finite element method, captures a variety of needle tip geometries, and allows generic needle shape manipulation inputs that go beyond base manipulation. We then conduct thorough analysis on simulation-driven control and feedback methods, particularly resolved-rate control for base manipulation with symmetric-tip needles, and a comparison of Broyden's update versus finite difference methods for obtaining a system Jacobian to manipulate the shape of bevel-tip needles. We finish \added[id=ii]{in~\cref{chap:chap-5}} by presenting sampling-based trajectory generating and tracking methods, and verify our proposed approach using both plastisol and \textit{ex viv} soft tissues.

Throughout this work, our model reside in the plane of adjustment, and tissue deformation is not explicitly solved using finite element method. Our design choice is particularly influenced by our early focus on MRI-guided spinal injections, during which the spinal target does not move and tissue motion cannot be efficiently tracked. Here, we would like to highlight some additional considerations when it comes to modeling, simulation, and control of flexible needles:

Model complexity, speed of solution, and feedback modality go hand-in-hand. Complex models tend to take longer to solve, which might hinder time-sensitive decision making during insertion. Yet simple models require higher-fidelity feedback to account for model deficiencies. For example, coupling a ``virtual spring'' model with fluoroscopy can achieve fast and accurate needle insertion, yet the cost of radiation is high; a more complex needle-tissue interaction model can achieve a higher simulation fidelity, but requires more computational overhead to obtain a solution. High-fidelity models often need to undergo model reduction to be suitable for control, therefore there needs to be a balance between model complexity and rate of solution to ensure a timely and good sim-to-real transfer.

Some simulators take target motion into account. This is useful for offline simulation, and possibly online feedforward control. However, due to the inevitable sim-to-real gap, questions remain whether the target motion can be effectively tracked, and whether the online controller can adjust to account for such discrepancy. In our current work, we do not rely on medical imaging for robot control, therefore inherently limits our ability to track target motion; however, as discussed in~\cref{sec:chap-5-results-and-discussion}, if target state can be tracked, we can augment the system state so that the same control framework can be used.

In this work, our attention has been focused on fully autonomous needle control, with the intention to push the boundary of surgical automation and empower future generations of clinicians. However, we recognize the importance of human guidance, which inspired this work in the first place. It is only by gaining their approval and trust can we develop robust cooperation between surgeons and robots, and bring beneficial changes to the healthcare landscape.


\section{Limitations}
\label{sec:chap-6-limitations}

\added[id=rt]{In our needle-tissue interaction model, tissue force is a function of stretch ratio $\lambda$, which is defined as $\lambda=\frac{ti - \vert u(x) \vert}{ti}$, which is an even function with respect to $u(x)$ to ensure the tissue force is symmetric regardless of direction of compression. This assumption does not hold when the soft tissue properties around the needle are different, and that tissue forces generated by the same amount of compression are different. The model also does not consider cases where relative motion between tissue layers are present, as is often the case for liver and prostate biopsies. When there is relative tissue motion, not only does tissue forces differ around the needle, such motion can also deflect the needle if the needle is inserted into these regions.}

\added[id=rt]{As mentioned previously, while these complexities could be captured by more sophisticated simulators, the ability to \emph{sense} sim-to-real discrepancies, as well as to \emph{control} the needle to adapt to these changes, is perhaps more important for intraoperative use. One has to balance model fidelity, speed of solution, and integration of feedback in order to make simulation-driven control framework more useful for clinical applications.}

\added[id=rt]{The choice of creating a planar simulator is based in part by the fact that, realtime medical imaging (e.g. X-ray fluoroscopy) are 2D, and the optimal angle for imaging both the needle and the target is perpendicular to the plane of adjustment, so that intraoperative imaging \emph{can} be used as needle positional feedback to the controller. If the same modeling strategy were to be used to develop a 3D simulation, the simulator would simply involve more states to represent the needle, and the needle bevel direction would be in $\mathbb{S}^1$. Additionally, the robot embodiment would involve much more complex arrangement to fully create the required DOF. In this dissertation, the focus is not solely on building a simulator, but to explore ways our simulation-driven framework can be used for image-agnostic needle control. However, we recognize the inherent shortcomings of using a planar simulator (e.g. out-of-plane error cannot be accounted for), and the additional challenges in both model formulation and hardware setup for a fully 3D setup.}

\added[id=ii]{Throughout this work, stainless steel needles of gauges between 18 and 22 are used. During experimentation, we found that these clinical needles exhibit a balance from both flexural compliance as well as flexibility that make them amenable to \textit{in situ} manipulation. Stiffer needles (e.g. bone biopsy needles are typically 11 gauges) will experience much less deflection due to soft tissue forces regardless of tip geometry, and softer needles (e.g. nitinol wires used for nonholonomic needle models) are less susceptible to needle base or shape manipulations. We expect our methodologies will apply to percutaneous interventions where the needle gauge falls between 18 and 22, and where the target stays relatively stationary. However, if target tracking via intraoperative imaging is available, we expect our framework to work as well.}

\added[id=rt]{Lastly, in order to remove the reliance on medical imaging during needle insertion, methods of needle feedback used in this dissertation will limit the actual clinical functionality of the needle. For example, to provide needle tip position feedback, an EM coil needs to be firmly embedded inside the needle, which make it difficult to perform procedures such as injection or biopsy. When using FBG fibers as sparse shape feedback, even though the fibers are attached to the inner stylet of the needle and can be removed when retracting the stylet, the added complication of instrument sterility as well as the cost of needle fabrication reduces its clinical viability.}

\section{Future Work}
\label{sec:chap-6-future-work}

\added[id=rt]{The simulation-driven framework introduced in this work has demonstrated significant potential for advancing medical robotics and surgical automation, particularly in image-agnostic percutaneous needle interventions. However, critical challenges remain in translating these advancements into robust clinical practice. To bridge this gap, the following research directions are proposed to extend the foundational contributions of this dissertation and inspire future innovation:}

\added[id=rt]{Physics-Aware Digital Twin Development for Surgical Manipulation: A systematic investigation into modeling safe surgical maneuvers and simulating tissue-tool interactions as digital twins will be essential. Future efforts could focus on extracting clinically validated motion primitives from procedural data and grounding them in physics-based models to enable real-time safety reasoning. Developing high-fidelity simulations of tissue mechanics and tool dynamics may provide insights into optimizing operational safety while preserving computational efficiency.}

\added[id=rt]{Intelligent Surgical Device Design and Workflow Integration: The development of next-generation surgical tools will require embedding low-cost, biocompatible sensing modalities to enhance intraoperative feedback. Research should explore how sensor data can be fused with simulation outputs to enable context-aware decision-making during procedures. Additionally, designing ergonomic human-robot collaboration frameworks—balancing surgeon expertise with autonomous assistance—could redefine clinical workflows to improve patient outcomes and reduce cognitive burden on practitioners.}

\added[id=rt]{AI-Driven Surgical Automation and Simulation-Guided Education: Further exploration of synergies between surgical simulators and AI technologies could unlock adaptive automation strategies for improving precision and efficiency. This includes leveraging simulation-generated datasets to train robust, generalizable algorithms for data-driven surgical planning. Concurrently, research into AI-powered training platforms may transform surgical education by enabling personalized skill assessment, procedural rehearsal, and competency-driven feedback loops.}

\added[id=rt]{Addressing these challenges will necessitate multidisciplinary collaboration across robotics, materials science, biomechanics, and clinical medicine. Translational research efforts must prioritize validating these technologies in realistic surgical environments while addressing ethical, regulatory, and practical barriers to adoption. Ultimately, such advancements hold the potential to redefine standards of care in minimally invasive interventions while fostering safer, more accessible surgical solutions.}
%%% Local Variables:
%%% mode: latex
%%% TeX-master: "main"
%%% End:
