\chap{Abstract} 


%%%% your abstract goes here (word limit: 350)

Image-guided percutaneous needle interventions are minimally invasive procedures that use intraoperative medical imaging to assist the insertion of needles through the skin for diagnosis and treatment. While modern robotic systems are capable of positioning the needle to the skin entry point, the actual insertion has mostly been relegated to the surgeons due to complex needle-tissue interactions that deviate the needle from preoperative surgical plans. \textit{In situ} needle manipulations are often necessary to correct initial needle misplacement prior to operation (e.g. biopsy, ablation, injection), which necessitates additional imaging and experience from the surgeons.

This thesis presents a suite of image-agnostic methods to perform fully autonomous needle manipulations during needle insertion. Taking a model-first approach, we present a mechanics-based needle-tissue interaction model inspired by \textit{in situ} needle manipulation motions from surgeons, and extends the model to a real-time finite element simulator that captures a variety of needle types as well as a variety needle manipulation inputs. Using the simulator as a physics engine, we investigate methods for flexible needle control and path planning, and how image-free feedback modalities can be used to provide online model correction for closed-loop control.

%% list of keywords seperated by comma
\keywords{Minimally Invasive Surgery, Continuum Instrument Control}


%%%%  committee members (add it right after the abstract w/o page break)
\begin{singlespace}

    %% if you have co-advisor, add here w/ \vspace{0.1in} as shown below
    %% alternatively you can use minipage environment to put side-by-side
    \section*{Primary reader and thesis advisor}
    
    Dr. Iulian Iordachita \\
    Professor\\
    Department of Mechanical Engineering\\
    Johns Hopkins University, Baltimore MD 


    \section*{Secondary readers}
    
    Dr. Russell Taylor\\
    Professor\\
    Department of Computer Science \\
    Johns Hopkins University, Baltimore, MD 
    
    \vspace{0.1in}
    
    Dr. Axel Krieger \\
    Professor\\
    Department of Mechanical Engineering \\
    Johns Hopkins University, Baltimore, MD 

    %% you can add more readers if you have them on your committee 
    %% use \vspace{0.1in} in between members for clarity
    %% you can also place committee members side-by-side using `minipage`


\end{singlespace}
%%% Local Variables:
%%% mode: latex
%%% TeX-master: "main"
%%% End:
