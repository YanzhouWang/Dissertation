\chap{Abstract} 


%%%% your abstract goes here (word limit: 350)

% Image-guided percutaneous needle interventions are minimally invasive procedures that use intraoperative medical imaging to assist the insertion of needles through the skin for diagnosis and treatment. While modern robotic systems are capable of positioning the needle to the skin entry point, the actual insertion has mostly been relegated to the surgeons due to complex needle-tissue interactions that deviate the needle from preoperative surgical plans. \textit{In situ} needle manipulations are often necessary to correct initial needle misplacement prior to operation (e.g. biopsy, ablation, injection), which necessitates additional imaging and experience from the surgeons.

% This thesis presents a suite of image-agnostic methods to perform fully autonomous needle manipulations during needle insertion. Taking a model-first approach, we present a mechanics-based needle-tissue interaction model inspired by \textit{in situ} needle manipulation motions from surgeons, and extends the model to a real-time finite element simulator that captures a variety of needle types as well as a variety needle manipulation inputs. Using the simulator as a physics engine, we investigate methods for flexible needle control and path planning, and how image-free feedback modalities can be used to provide online model correction for closed-loop control.

Image-guided percutaneous needle interventions are minimally invasive medical procedures that utilize intraoperative medical imaging to guide needle insertion through the skin for diagnostic and therapeutic purposes. While modern robotic systems can accurately position the needle at the skin entry point, the insertion process remains predominantly surgeon-dependent due to complex needle-tissue interactions that frequently cause path deviations from preoperative plans. Such deviations often necessitate real-time \textit{in situ} needle adjustments prior to the surgical operation (e.g., biopsy, ablation, or injection), requiring repeated intraoperative imaging and significant clinical expertise.

This thesis introduces a suite of image-agnostic methods for fully autonomous needle manipulation during insertion. Adopting a model-driven approach, \replaced[id=ak]{I}{we} first present a mechanics-based needle-tissue interaction model inspired by surgeons' freehand correction techniques. This model is extended into a real-time finite element simulator capable of replicating diverse needle types and manipulation strategies. Leveraging the simulator as a physics engine, \replaced[id=ak]{I}{we} develop novel solutions for needle control and path planning, while demonstrating how image-free feedback modalities can facilitate online model reconstruction for closed-loop control. By bridging simulation with real-world dynamics, this work advances autonomous needle steering systems that reduce reliance on intraoperative imaging and operator skill.

%% list of keywords seperated by comma
\keywords{Minimally Invasive Surgery, Finite Element Simulation, Control}


%%%%  committee members (add it right after the abstract w/o page break)
\begin{singlespace}

    %% if you have co-advisor, add here w/ \vspace{0.1in} as shown below
    %% alternatively you can use minipage environment to put side-by-side
    \section*{Primary reader and thesis advisor}
    
    Dr. Iulian Iordachita \\
    \added[id=ii]{Research} Professor\\
    Department of Mechanical Engineering\\
    Johns Hopkins University, Baltimore MD 


    \section*{Secondary readers}
    
    Dr. Russell Taylor\\
    Professor\\
    Department of Computer Science \\
    Johns Hopkins University, Baltimore, MD 
    
    \vspace{0.1in}
    
    Dr. Axel Krieger \\
    \added[id=ak]{Associate} Professor\\
    Department of Mechanical Engineering \\
    Johns Hopkins University, Baltimore, MD 

    %% you can add more readers if you have them on your committee 
    %% use \vspace{0.1in} in between members for clarity
    %% you can also place committee members side-by-side using `minipage`


\end{singlespace}
%%% Local Variables:
%%% mode: latex
%%% TeX-master: "main"
%%% End:
