%%%% necessary customization for the appendix and headers
\appendix 
\makeatletter
\addtocontents{toc}{\protect\renewcommand\protect\cftchappresnum{\@chapapp\ }}
\makeatother
\renewcommand{\thechapter}{\Alph{chapter}}

\chapter{Test Specimen Fabrication} \label{chap:appendix-a}
\begin{figure}[h]
  \centering
  \includegraphics[width = 0.68\columnwidth]{manuscripts/RAL_2023/figures/specimen_equipment.png}
  \caption{Equipment used to produce cylindrical test samples. A honed tube is used both for preforming the sample and also for treating the uneven cut surfaces. A heating plate transfers heat from a digital hotplate. A pushrod ensures contact between the uneven cut surface and the heating plate during surface treatment. Cylindrical specimens that are compliant to test standard can be fabricated with high repeatability.}
  \label{fig:specimen_equipment}
\end{figure}

Plastisols have been frequently used to make transparent, rubber-like tissue phantoms for modeling needle-based applications, but few have reported fabrication methods for test specimens for the purpose of obtaining their mechanical properties. Here, we present our method for fabricating cylindrical test specimens from liquid plastisols with high repeatability, and the test specimen fabricated can be compliant to compression testing standards.

A major difficulty for fabricating cylindrical test specimens is to create surfaces that are smooth. Due to high flexibility of these materials, cutting them with a blade does not produce a flat surface, as they will deform as soon as there is pressure applied. Here, we use the fact that cured plastisols melt under high temperature in order to treat the uneven cut surfaces.

In terms of equipment, a honed tube with 1-1/8in inner diameter (ID) with two slightly enlarged ends is used to form the cylindrical shape of the specimen. A heating plate can be securely attached to either side of the tube, and is used to transfer heat from a digital hotplate. The ID for surface treatment (green) is slightly larger than that of the preforming side of the tube, allowing easier attachment of the tube with the heating plate. A pushrod is used to ensure contact between the preformed specimen with the heating plate during the surface treatment process. Temperature and time listed are specific to the plastisol composition used in this study.

\begin{enumerate}
\item Pre-heat the digital hotplate to 325$^\circ$C.
\item Measure 50ml liquid plastisol.
\item Attach the heating plate to the preforming side of the tube.
\item Pour the measured plastisol in the tube and heat for 30mins; shake the tube at 15min time stamp.
\item Transfer tube into a liquid bath that submerges the bottom section of the tube, and let it cool for 15 mins.
\item Take the tube out of the bath, and remove the heating plate by pulling straight. The preformed sample should be attached to the inner wall of the tube, and detached from the heating plate.
\item Peel the sample off of the inner wall of the tube and remove from the tube.
\item Cut the sample to a desired length with a blade. 
\item Reheat only the bottom plate to 300$^\circ$C, and transfer it onto a workbench.
\item Re-insert the sample to the surface treatment side of the tube, with the uneven surface facing the heating plate. 
\item Re-attach the tube to the heating plate, and use the pushrod to generate pressure on the sample to ensure contact between the uneven surface and the heating plate. Wait for 2mins.
\item Transfer the assembly to the liquid bath, and submerge the bottom section for 10mins.
\item Remove the assembly from the liquid bath, and remove the heating plate.
\item Peel the finished sample off of the inner wall of the tube. Process complete.
\end{enumerate}

%%% Local Variables:
%%% mode: LaTeX
%%% TeX-master: "main"
%%% End:
